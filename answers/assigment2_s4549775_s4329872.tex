\documentclass[12pt]{article}

\usepackage{listings}
\usepackage[top=5em, bottom=5em, left=5em, right=5em]{geometry}

\setlength\parindent{0em}
\setlength\parskip{1em}

\title {Assignment 2}

\author {Hendrik Werner s4549775}

\begin{document}
\maketitle

This was done in collaboration with Constantin Blach (s4329872).

\section{} %1
Here is an implementation of a Queue using two stacks to store its elements written in Groovy:

\lstinputlisting{code/TwoStackQueue.groovy}

It can be used like this:

\begin{lstlisting}
def range = 1..10
TwoStackQueue<Integer> q = new TwoStackQueue()
for (int i in range) {
    q.add i
}
for (int i in range) {
    println q.get()
}
\end{lstlisting}

\section{} %2
Here is an imlpementation of a Deque using an array to store its elements written in Groovy:

\lstinputlisting{code/ArrayDeque.groovy}

Here is a usage example:

\begin{lstlisting}
def deque = new ArrayDeque<Integer>(2)
deque.addEnd(3)
deque.addStart(5)
println deque.removeStart()
println deque.removeEnd()
\end{lstlisting}

\section{} %3
Here is an implementation of a singly linked, circular list in written in Groovy:

\lstinputlisting{code/SinglyLinkedCircularList.groovy}

You could implement a method to delete the next element in $O(1)$. You could also implement a method to look at the next element in $O(1)$ to be able to use the $deleteNext$ method in a useful way.

\section{} %4
\section{} %5
\section{} %6

\end{document}
